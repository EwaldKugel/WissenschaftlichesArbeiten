\documentclass[fontsize=11, titlepage=true, parskip=half]{scrartcl}
\usepackage{geometry}			%Papierformat und Rand
\geometry{a4paper, left=2cm, right=2cm, top=2cm, bottom=2cm}
\usepackage[utf8]{inputenc}
\usepackage[ngerman]{babel}
\usepackage{amsmath}
\usepackage{amssymb}
\usepackage{pdfsync}
\usepackage{pdflscape}			%landscape Umgebung für Querformatanzeige
\usepackage{siunitx}			%schöne SI Einheiten
\usepackage{graphicx}
%\usepackage{abstract}			%??
%\usepackage{microtype}		%saubere Umbrüche bei mehrspaltigen Docs
\usepackage{booktabs}			%??
\usepackage[hyphens]{url}
\usepackage{eurosym}			%schönes Euro Symbol
%\usepackage{dblfloatfix}		%float bottom twocolumn
\usepackage[hidelinks]{hyperref}	% Paket: ltxcmds kann Links stylen
\usepackage[automark]{scrlayer-scrpage} 	%Kopf und Fußzeile bearbeiten Paket: koma-script
\chead*{\pagemark} %Seitenzahl mittig in Kopfzeile
\cfoot[]{} %Voreinstellung für Seitenzahl löschen
\usepackage{color}				%eigene Farben definieren (für Code-Highlighting)
\usepackage[comma]{natbib}
\usepackage[onehalfspacing]{setspace}
\newcommand{\csharp}{{\settoheight{\dimen0}{C}C\kern-.05em \resizebox{!}{\dimen0}{\raisebox{\depth}{\#}}}} %schönes C#

%Farben für C#
\definecolor{bluekeywords}{rgb}{0,0,1}
\definecolor{greencomments}{rgb}{0,0.5,0}
\definecolor{redstrings}{rgb}{0.64,0.08,0.08}
\definecolor{xmlcomments}{rgb}{0.5,0.5,0.5}
\definecolor{types}{rgb}{0.17,0.57,0.68}

%Farben für HTML


%Listings für Code
\usepackage{listings}
\lstset{frame=lines, % Oberhalb und unterhalb des Listings ist eine Linie
showstringspaces=false,
basicstyle=\ttfamily\small,
showspaces=false,
showtabs=false,
breaklines=true,
breakatwhitespace=true,
escapeinside={(*@}{@*)},
}

\lstdefinestyle{CSharp}{language=[Sharp]C,
captionpos=b,
%numbers=left, %Nummerierung
%numberstyle=\tiny, % kleine Zeilennummern
commentstyle=\color{greencomments},
morekeywords={partial, var, value, get, set},
keywordstyle=\color{bluekeywords},
stringstyle=\color{redstrings},
}

\lstdefinestyle{HTML}{language=HTML,
% sensitive=true,
% classoffset=0,	
% stringstyle=[0]\color{bluekeywords},
% keywordstyle=[0]\color{redstrings},
% classoffset=1,
% morekeywords={
% %Angular
% [(ngModel)], (change), *ngIf,
% },
% keywordstyle=[1]\color{red},
}

\AfterTOCHead{
\singlespacing
\thispagestyle{empty}
}



% \includeonly{
% Titelblatt2,
% Abstract,
% Einleitung,
% Problembeschreibung,
% Grundlagen,
% Analyse_des_Problems,
% Lösung,
% Zusammenfassung_und_Diskussion,
% Fazit_und_Ausblick,
% }

\begin{document}
\pagenumbering{roman}
\titlehead{
{\Large Hochschule Rhein Waal}\\
Fakultät für Kommunikation und Umwelt\\
Friedrich-Heinrich-Allee 25\\
47475 Kamp-Lintfort
}

\subject{
Abschlussbericht\\
\normalfont
im Modul Wissenschaftliches Arbeiten\\
}

\title{Super Duper Realthema}

\subtitle{Untertitel für super Realthema}
%TODO Timo: Matrikelnummer eintragen
\author{
Dario Becker
\thanks{
 Matr.-Nr.: 28492\\
E-Mail: dario.becker@hsrw.org}
\and Timo Hungenberg
\thanks{
Matr.-Nr.: XXX\\
E-Mail: timo.hungenberg@hsrw.org}
}

\date{\today}

% \publishers{
% Betreut durch\\
% Tobias Scharlewsky
% \thanks{
% Telefon: +49(203)41752945\\
% E-Mail: tobias.scharlewsky@polizei.nrw.de}\\
% Prof. Dr. Thomas Richter}

\maketitle


\tableofcontents

\cleardoubleoddpage
\pagenumbering{arabic}
%\KOMAoption{parskip}{half}

\section{Einleitung}
% Hier Deepfakes im allgemeinen Vorstellen: Welche Arten gibt es und wie werden die verbreitet?
% Aktuelles Thema als Einführung in unsere konkrete Problematik: Von Deepfake allgemein zu Teilaspekt führen den wir besprechen.
% Enden mit Forschungsfrage bzw. Ziel des Artikel: Was wollen wir zeigen, worauf wollen wir hinaus?
In der heutigen Zeit nimmt das Aufkommen und die Verbreitung von Falschmeldung zu \citep[][]{Hancock2021}.
Im Besonderen mehren sich die mit Hilfe von künstlicher Intelligenz erschaffenen manipulierten Falschmeldungen, sogenannte Deepfakes \citep[][]{Shahzad2022}.
Der Begriff des Deepfake entsteht aus ``Deeplearning'', einer auf künstlicher Intelligenz basierenden Methode des maschinellen Lernens, und ``Fake'' welcher mit Hilfe dieses ``Deeplearnings'' erstellt wird und den Menschen täuschen soll \citep[][]{Mueller2022}.
Innerhalb der Deepfakes werden z.B. Gesichter in eine Bild- oder Videodatei realistisch geschnitten, um diese Personen beliebige Worte sagen zu lassen.
So zeigt ein aktuelles Beispiel, wie Olaf Scholz eine angebliche Rede hält in der er über russische Gaslieferungen spricht \citep[][]{Klasen2022}.
Über Twitter wird die Reaktion Putins auf diese Rede von einer russischen Nachrichtenagentur geteilt \citep[Vgl.][]{Klasen2022}.
Gerade über die Kanäle der sozialen Medien wie Twitter oder Facebook, lassen sich diese Falschmeldungen heutzutage schnell und gezielt verbreiten (siehe Putin-Scholz Beispiel).
\par
Eine weitere Form des Deepfakes ist die Manipulation von Audiodateien (Audio Deepfakes, AD).
Diese Art Deepfake wird oftmals von Betrügern genutzt, um potentielle Opfer am Telefon oder in Interviews zu täuschen, also in Echtzeit \citep[][]{Mueller2022}.
Wie bereits einige Fälle gezeigt haben \citep[vgl.][]{Stupp2019}, nimmt die zeitliche Komponente bei der Erkennung dabei eine besondere Rolle ein.
Denn ist der Deepfake einmal geteilt und in den Köpfen der Menschen, lassen sich Falschmeldungen nur schwierig korrigieren \citep[][]{Hancock2021}.
Dabei ist die nachträgliche Abwendung der Gefahr von Echtzeit-, also oftmals AD, dabei weitaus schwieriger als auf sozialen Medien geteilten Video- oder Bildmaterialien \citep[][]{Shahzad2022}.
\par
% Um Deepfakes als soche zu Erkennung, werden in der Literatur einige Merkmale wie z.B. die Asynchronität von Stimme und Lippenbewegung genannt \citep[][]{Appel2022}.
% Auch das zu häufige Blinzeln, Kopfhaltung oder Falschstellungen von Zähnen, können ein Indiz auf einen Deepfake sein \citep[][]{Shahzad2022}.
Da die menschliche Wahrnehmung zur Erkennung eines Deepfakes allerdings limitiert ist, empfehlen Wissenschaftler den Einsatz von technischen Hilfsmitteln \citep[][]{Mueller2022}.
Zwar wird sich für Maßnahmen ausgesprochen um gegen Deepfakes zu sensibilisieren, aber das wird zukünftig nicht reichen da die Technologie zur Erstellung solcher Deepfakes immer besser wird \citep[][]{Amezaga2022}.
% Daher ist es unbedingt notwendig, dass die Erkennung mit technischen Hilfsmitteln weiterentwickelt werden muss.
% Es gibt einige technische Lösungen zur Erkennung von Audio-Deepfakes, die ähnlich wie die Erstellung von Deepfakes auf künstlicher Intelligenz basieren, welche aber den zeitlichen Faktor nicht berücksichtigen.
\par
In dieser Arbeit wird aufgezeigt, welche Gefahrenpotentiale AD beinhalten, wie die menschliche Wahrnehmung zu solchen Fakes ist und wie die Technik dabei helfen kann diese in Echtzeit zu entlarven.
Dabei werden verschiedene Methoden zur Erstellung und Erkennung solcher Fakes analysiert und diskutiert.
Vor allem werden die beiden Methoden Text-to-Speech und Voice Conversion, welche aktuell überwiegend zur Erzeugung von AD genutzt werden, genauer untersucht.
Die Frage, ob und in welchem Grad AD mit Hilfe aktueller technischer Lösungen in Echtzeit erkennbar sind, wird in mehrere Herausforderungen unterteilt, analysiert und beantwortet.
Dabei ist das Ziel dieser Fragestellung, welche Art von AD mit welchen Ansätzen der technischen Erkennung in welcher Zeit zu erkennen sind und wie die davon abgeleitete zukünftige Behandlung solcher Fakes zu gestalten ist. 

\section{Gefahrenanalyse}
% Gesellschaftliche Betrachtung der Thematik.
% 2-3 Sätze die erklären, was im Kapitel passiert.
In diesem Kapitel werden die Angriffspunkte von Deepfakes und den damit verbundenen Gefahren für die Gesellschaft beschrieben.
Anhand einiger Beispiele wird aufgezeigt, welchen Einfluss Deepfakes haben, um beim Menschen durch gezielte Manipulation Schaden anzurichten.
Weiterhin werden Möglichkeiten zur schnelleren Erkennung solcher Manipulationen diskutiert.
\subsection{Gefahrenpotential allgemein}\label{Gefahrenpotential}
% Hier zunächst allgemeine Gefahrenpotentiale von gezielter Fehlinformation definieren.
Das Konzept der Falschinformation, in der heutigen Zeit oft unter dem Begriff ``Fake News'' genannt, ist eine Methode die bereits seit mehr als 125 Jahren eingesetzt wird um Menschen gezielt zu manipulieren.
Gerade im Zeitalter der sozialen Medien, in denen sich Menschen öfter dieser Meldungen bedienen als konventioneller Nachrichten, wächst die Bedrohung durch Falschinformation täglich (\cite{Lee2019}).
Eine große Gefahr die dabei von Deepfakes ausgeht, ist dass jeder die Möglichkeit hat, diese mit kostenfreier Software zu erstellen (\cite{Appel2022}).
Das gefährliche dabei ist die millionenfache Verbreitung innerhalb Sekunden um die ganze Welt, in der Jeder Ziel eines solchen Angriffs sein kann (\cite{Shahzad2022}).
\par
Dabei nennen Appel und Prietzl (\cite{Appel2022}) Angriffsziele von privaten Personen, über Prominente Personen bis hin zu ganzen Staaten.
Besonders letzteres ist ein immer häufiger gewähltes Ziel, gerade bei politschen Wahlen oder Zwischenstaatlichen Beziehungen.
So kursierte im März 2022 in sozialen Medien ein Deepfake-Video des Ukrainischen Präsidenten Volodymyr Zelenskyy, in dem er im Rahmen des Russisch-ukrainischen Krieges die Kapitulation seiner Soldaten forderte.
Einmal durch die Kanäle sozialer Medien verbreitet, lassen sich diese Falschmeldungen schwierig Löschen oder Richtigstellen und wenn doch, dann ist der Schaden bereits angerichtet.
Soziale Netzwerke oder Messenger wie Facebook, WhatsApp oder Telegram sind dabei besonders kritisch zu betrachten, da sie die Verbreitung von ungeprüften Inhalten jeglicher Art ermöglichen (Vgl. \cite{Appel2022}).
\par
Deepfakes erweitern und verschlimmern also das bereits bekannte Phänomen der ``Fake News'', indem sie sich der gleichen Kanäle bedienen, die Inhalte allerdings noch deutlich realistischer und glaubhafter darstellen (\cite{Appel2022}).
Gerade das hohe Level von Faktoren wie Verbreitungsgeschwindigkeit, Realismus und Personalisierung machen Deepfakes zu einer immer weiter wachsenden Gefahr für die Gesellschaft.
Was den Menschen dabei so angreifbar macht, ist die Tatsache, dass vermeintlich selbst wahrgenommen Inhalte wie Videos und Fotos, einen stärkeren Eindruck hinterlassen als niedergschriebener Text.
Besonders wenn es sich dabei um vertraute Personen und deren Auftreten oder Stimme handelt (\cite{Kietzmann2020})
\newpage

\subsection{Gefahrenanalyse am Beispiel aktueller Fälle}\label{GefahrenAktuelleFaelle}
% Gefahrenpotentiale an Beispielen konkretisieren: \textbf{Was} wurde gemacht um \textbf{was} zu erreichen?
% Hier Fokus auf Echtzeitproblematik legen? Oder folgt das aus dem nächsten Unterkapitel?
Folgend werden einige aus dem unter Abschnitt \ref{Gefahrenpotential} beschriebenen Gefahren anhand konkreter Realfällen diskutiert. 
Bei diesen Fällen handelt es sich hauptsächlich um Betrugsfälle, die mit Hilfe von Deepfakes durchgeführt wurden.
\par
Einer dieser Fälle, der im Zusammenhang mit Deepfake-Betrug genannt wird, ist die Erbeutung von 220.000\euro{} durch Verwendung eines Audio-Deepfakes.
In diesem Fall stellten Betrüger die Stimme eines CEO am Telefon nach und baten seinen angeblichen Mitarbeiter, den genannten Betrag auf ein Konto zu überweisen \citep[][]{Stupp2019}. 
Der angerufene Mann berichtete nachträglich an die Ermittlungsbeamten, dass er den deutschen Akzent und die Melodie der Stimme seines CEO's erkannte, und somit keinen Verdacht schöpfte dass es sich bei diesem Anrufer um einen Betrüger handelte.
Die Ermittlungen gingen davon aus, dass eine kommerzielle Software zur Erstellung des Deepfakes verwendet wurde.
Dieses Beispiel verdeutlicht, dass jeder mit entsprechender freizugänglicher Software im Stande ist, so einen Betrug mit Hilfe eines Audio-Deepfakes durchzuführen \citep[Vgl.][]{Stupp2019}. 
\par
Eine weitere Betrugsmasche, vor der das FBI offiziell warnte, sind Betrugsfälle in denen sich mit durch Video Deepfake veränderter Stimme für sensible Jobs beworben wird.
Während dieser Jobinterviews, die meist auf Jobs mit Heimarbeit in Softwareunternehmen mit großen Datenmengen abzielen, benutzen Betrüger die Stimme einer anderen Person.
Das FBI betonte hierbei aber, dass die Synchronisation zwischen Lippenbewegung und Sprache nicht komplett übereinstimmte.
Somit gibt es Anhaltspunkte zur Vorbeugung und Erkennung solcher Betrüge in Form von Jobinterviews, indem Mitarbeiter von Unternehmen geschult werden um unter anderem auf solche Merkmale verschärft zu achten \citep[][]{Ferraro2022}. 
\par
Zur Veranschaulichung der potentiellen politschen Manipulation unter der Verwendung von Deepfakes, ist die Rede der amerikanischen Politikerin Nancy Pelosi aus dem Jahr 2019 zu nennen \citep[Vgl.][]{Mervosh2019}.
In dieser Rede wurde das Videomaterial so manipuliert, dass es so scheint als ob sie stotterte und undeutlich rede.
Das manipulierte Video wurde von dem zu dieser Zeit amerikanischen Staatspräsidenten und Anhänger der Gegnerpartei von Nancy Pelosi, Donald Trump, über Twitter verbreitet.
Er teilte es mit den Worten ``Pelosi stammers through news conference'', also mit der gezielten Absicht sie mit Hilfe dieses Videos zu diffamieren \citep[][]{Mervosh2019}.
\par
Dieses Video zeigt die Gefahr der schnellen Verbreitung von ungeprüften Inhalten durch soziale Medien, die unter Abschnitt \ref{Gefahrenpotential} genannt wurde.
Über Facebook wurde dieses Video über 2.5 Millionen mal angeschaut, Facebook selbst lies dieses Video auf der Plattform bestehen versprach aber die Verbreitung einzugrenzen.
Youtube hingegen löschte dieses Video, da es sich um falsche Inhalte handelte.
Dieses unterschiedliche Behandeln von Falschinformation verschiedener sozialer Medien zeigt die Schwierigkeit des Konsums von Informationen über diese Plattformen.
Es bleibt daher ein beliebtes Mittel zur Verbreitung solcher Inhalte, eben aufgrund der freien Verbreitung, Geschwindigkeit und umständlicher Löschung dieser \citep[][]{Appel2022}.
\par
Die hier beschriebenen Betrugsfälle veranschaulichen, dass wie unter Abschnitt \ref{Gefahrenpotential} beschrieben, mit einfachsten Mitteln großer Schaden angerichtet werden kann.
Besonders die von Kietzmann beschriebene Vertrautheit von Inhalten wie z.B. die Stimme im Zusammenspiel mit Gestik und Mimik, ließ bei den erwähnten Beispielen keinen Zweifel an der Echtheit \citep[][]{Kietzmann2020}.
So reichte es bei dem Telefonbetrug, dass der deutsche Akzent und die Melodie der Stimme vermeintlich übereinstimmte, keinen Verdacht zu schöpfen.
Darüberhinaus zeigen die genannten Beispiele, dass die Echtzeit und damit die nahezu nicht vorhandene Zeit zur Erkennung eine entscheidene Rolle spielt.
Einmal verbreitet lässt sich der Schaden nur erschwert beheben oder die Inhalte schwierig Löschen oder Richtigstellen \citep[][]{Shahzad2022}.
Dabei betonte Shahzad einmal mehr, zukünftig effektive Methoden zur Erkennung in Echtzeit zu entwickeln, da diese den Großteil der Bedrohung darstellen.
Ein weiteres Gefahrenpotential bleibt der einfache Zugang zur Erstellung solcher Deepfakes.
Mit den richtigen Zugängen zu Verbreitungskanälen und anzusprechenden Opfern, ist es jedem Menschen möglich das Mittel der Manipulation durch Deepfakes zum Betrug einzusetzen \citep[][]{Appel2022}.




\subsection{Gefahrenanalyse/Studien zur menschlichen Wahrnehmung}
Analyse von durchgeführten Studien: Inwieweit ist der Mensch anfällig für Deepfakes?
Welche Konsequenz ergibt sich daraus? (Anforderungen an Informationen, Detektion von DF, etc.)
Analyse von durchgeführten Studien: Inwieweit ist der Mensch anfällig für Deepfakes?
Welche Konsequenz ergibt sich daraus? (Anforderungen an Informationen, Detektion von DF, etc.)
\section{Technische Erkennung von Audio Deepfakes}
Analyse der verfügbaren Detektionsalgorithmen auf Effektivität und Effizienz im Hinblick auf Echtzeitproblematik
2-3 Sätze die erklären, was im Kapitel passiert.
\subsection{Methodische Ansätze}
Hier verfügbare Methoden kategorisieren und allgemein vorstellen.
Welche Ansätze gibt es?
Welche Voraussetzungen und Anforderungen haben die (Ressourcen, Quelle)?
\subsection{Effektivität und Effizienz}
Fokus auf Echtzeitproblematik: Was ist technisch möglich oder eben nicht?
Hier erstes kleines Fazit ziehen?
%\section{Recherche Dario}
Überblicksartikel über Chancen, Risiken, mögliche Lösungen.
Paper nicht veröffentlicht, daher besser die angegebenen Quellen nehmen wenn wir zitieren wollen.

Vorteile: Voice Assistant, educational content, Filmindustrie, Games.

Nachteile: Trump Beispiel um politischen Gegner zu diffamieren, China postet australischer Soldat hält Messer an die Kehle von afghanischem Mädchen: Verschlechterung diplomatischer Beziehungen
Business: Telefon Scam um Geld zu verschieben (Finanzsektor)

Lösung: Vorschlag digitale Signatur, Awareness, digital forensic, Regeln und Strafen (Gesetze): Angst vor Zensur. \citep{SAB2020}

Neuer Ansatz zur Überprüfung von Audio DF.
Leider auch keine publizierte Quelle, zumindest nicht klassisch.
Statt Detektor auf bestimmte DF Generatoren zu trainieren wird geprüft, ob die Person authentisch ist:
Detektor lernt auf realen Daten der echten Person und guckt dann mit Standard Mitteln zur Speaker Verification ob der angebliche Sprecher echt ist oder fake. \citep{Pianese2022}

Mega umfangreiches Übersichtspaper, geht einmal um alles.\citep{Masood2022}
%\section{Recherche Timo}
\subsection{How robust is the United Kingdom justice system against the advance of deepfake audio and video}
Dieses Paper untersucht, wie geschützt das Rechtssystem in UK vor Deepfakes ist.
Dabei werden Schwierigkeiten bei der Erkennung und vorbeugende Maßnahmen diskutiert.
Das Paper grenzt den Begriff Fakes von Deepfakes, wobei Fakes rein von menschlicher Hand, Deepfakes aber durch den Machine Learning Process erstellt werden.
Zur Vorbeugung nennt das Paper die Schulung und Sensibilsierung von Verantwortlichen in Gerichten zur einfacheren Erkennung von Deepfakes.
Gerade der Einsatz und die Schulung von technischen Forensikern wird empfohlen.
Zur Zeit gibt es noch keinen gesetzlichen Standard bzgl. Deepfakes in UK.\cite{Jones2022}

\subsection{Do (Microtargeted) Deepfakes Have Real Effects on Political Attitudes}
Umfrage mit 278 Teilnehmern, 5 Sekunden deepfaked an Video von Hollänischem Politiker christlicher Partei.
Studie zeigt dass es möglich ist zu manipulieren mit so einem Skandal. 
Das Ansehen sinkt aber eher an den Politker als an seine Partei, dahingehend gibt es nur kleine Auswirkungen.
Ergebnis zeigt dass Deepfakes einen höheren Einfluss haben als die Verbreitung von einfachen Falschmeldungen.
Nur 12 der Teilnehmer erkannten den Deepfake.\cite{Dobber2020}

\subsection{Human Perception of Audio Deepfakes}
Studie Über Erkennung von Audio Deepfakes von 410 Teilnehmern mit Hilfe eines Spiels, indem die Teilnehmer Stimmen hörten und diese als entweder Wahr oder Falsch deklarierten.
Muttersprachler höhere Erkennungschance als Nicht-Muttersprachler.
Ältere Menschen anfälliger als Junge Menschen.
IT-Kenntnis nicht von Bedeutung beim Erkennen von Deepfakes.
Aussicht ist dass die Technologie sich stark verbessern wird, die menschlichen Fähigkeiten zur Erkennung allerdings stagnieren.
Fazit. es braucht technische Erkennung. \cite{Mueller2022}

\subsection{A Review of Image Processing Techniques for Deepfakes}
Deepfakes erreichen Millionen Menschen innerhalb von Sekunden auf Social Image; Gefährlich.
Jeder kann Ziel eines Deepfake-Angriffs sein.
Ergebnis von Studien zeigt, dass man zur Bekämpfung dieser Berdohung die folgenden Maßnahmen ergriffen werden sollten:
Politische Richtlinien, Schulung und Bildung.
Jeder kann mit freier Software Deepfakes erstellen, kein besonderes Wissen nötig (z.B. Faceswap-Apps).
Paper untersucht verschiedene Ansätze von Erkennungen (BILD).
Unterscheidung zwischen Video, Audio und Tweet-Deepfakes.
Indikatoren zur Erkennung können sein: Kopfhaltung, Anzahl und Dauer von Blinzeln, Hinweise bei der Zahnplatzierung und weitere Gesichtszüge.
Voraussage: Deepfake wird immer weiter Verbreitung finden, vorallem auf Social Media.
\cite{Shahzad2022}

\subsection{FBI Warns That Scammers Are Using Deepfakes to Apply for Sensitive Jobs}
Gestohlene Identitäten für Jobinterviews für Remote-Arbeit mit Hilfe von Audio-Deepfake.
Als Gegenmaßnahme Schulungen für Mitarbeiter zur leichteren Erkennung.
Genutzt um Zugang zu Firmennetzen und Firmendaten zu erhalten.
Dabei sollte man auf die Synchronisation von Lippe und Stimme achten, oft nicht zu 100 Prozent übereinstimmend
Zur Vorbeugung Deepfake Risk Management einführen.
WIE ZITIEREN?
%\section{Expose}
\subsection{Grundlagen}
Bei Deepfakes handelt es sich um manipulierte Audio, Video und Bilddateien, die mit Hilfe von Deep Learning, also KI, erstellt werden.
Es gibt einige Beispiele zur Anwendung von Deepfakes, wie eine Videomanipulation, in der Barack Obama Donald Trump ``a total and complete dipshit'' nennt.
Gerade im politischen Sektor können Deepfakes genutzt werden um einen Gegner zu diskreditieren.
Diese Möglichkeit der politischen Dikreditierung zeigt eine Studie, in der die Auswirkungen auf eine manipulierte Videodatei eines holländischen Politikers untersucht werden.
Darüberhinaus zeigt die Studie, dass die Verwendung von Deepfakes eine höheren Einfluss hat als die Verwendung von klassischen Falschmeldungen (\cite{Dobber2020}).
Dieses Paper beschäftigt sich allerdings mit der Gefahr, die von Audio-Deepfakes ausgeht.
Es gibt bekannte Fälle, in denen sich sogenannte Scammer als CEO einer Firma ausgeben und als dieser eine Summe Geld einfordern (cite Onlinequelle Fake Johannes).
Der Schwerpunkt des Papers liegt bei der Gefahr, die von Audio-Deepfakes ausgeht, sowie die optimierte Erkennung von diesen.

\subsection{Aussagen zur Forschung}
Die Forschung, die auf diesem Gebiet betrieben wird, setzt sich häufig mit der Erstellung und Erkennung von diesen Deepfakes auseinander.
Bei der Untersuchung möglicher Identifikatoren wird zwischen der Erkennung durch den Menschen und Erkennung von Algorithmen unterschieden.
Die aktuelle Aussicht ist die, dass die Technologie zur Erkennung sich stark verbessern wird, die menschlichen Fähigkeiten allerdings stagnieren (\cite{Mueller2022}).
Hauptproblem bei der Erkennung von Deepfakes ist die zeitliche Komponente.
Gerade über soziale Medien erreichen Deepfakes innerhalb von Sekunden Millionen Menschen.
Studien zeigen potentielle Maßnahme zur Bekämpfung von Deepfakes wie politsche Richtlinien, Sensibilisierung, Schulung sowie Bildung (\cite{Shahzad2022}).
Auch wird der Einsatz von technischen Forensikern in sensiblen Gebieten wie vor Gericht empfohlen (\cite{Jones2022}).
Diese Technologie steht erst am Anfang und jeder ist in der Lage mit einfachster Software Deepfakes zu erstellen.
Der Fokus liegt zwischen der Erstellung und Erkennung von Deepfakes, der ``Kalte Krieg'' zwischen diesen beiden hat gerade erst begonnen (\cite{Masood2022})
\subsection{Forschungsnische}
Aktuelle Literatur zum Thema Deepfake und speziell Audio Deepfake kann grob in die Bereiche Risikobetrachtung und -untersuchung und technische Betrachtung aufgeteilt werden.
Bei der Risikobetrachtung werden Untersuchungen durchgeführt, die beispielsweise die Beeinflussbarkeit von Wählern durch Deepfakes  \citep[vgl.][]{Dobber2020} oder allgemein die Sensibilität von Bevölkerungsgruppen für Deepfakes untersuchen \citep[vgl.][]{Mueller2022}.
In der technischen Betrachtung gibt es aktuell eine Reihe von Übersichtsartikeln über diverse Detektionsmethoden für Audio Deepfakes \citep[vgl.][]{Masood2022,Khanjani2021,Almutairi2022}.
Hier liegt der Fokus darauf, verschiedene Methoden sowohl auf Effektivität als auch auf Effizienz zu überprüfen und Vorschläge für die weitere Forschung abzugeben.

Allen Artikeln ist dabei gemein, dass zunächst die Relevanz der Problematik an Beispielen oder theoretischen Überlegungen hervorgehoben wird.
Dabei gibt es durchaus auch Artikel, die auf positive Aspekte von beispielsweise Text-to-Speech (TTS) Technologien (Voice Assistant) oder auf harmlose Anwendungen von Deepfakes in der Filmindustrie eingehen.
Die Gefahren wie die Beeinflussung von politischen Wahlen, Telefonbetrug im teilweise großen Stil \citep[vgl.][]{} oder vergleichsweise kleiner Betrug im privaten Umfeld \citep[][]{} stehen jedoch bei allen Artikeln im Vordergrund.
Daraus wird hauptsächlich die Notwendigkeit der Forschung auf dem Gebiet der Deepfake Detection abgeleitet, nur wenige Artikel beschäftigen sich mit dem allgemeinen Vorgehen bzw. dem allgemeinen Umgang mit Deepfakes \citep[][]{}.

Da die Problematik wie oben gezeigt nicht nur alle Altersklassen sondern vor allem über gesellschaftliche Schichten, Berufe bis zu geopolitischen und zwischenstaatlichen Beziehungen so ziemlich jeden betrifft ist das ein besonders wichtiger Punkt.
Es muss die Frage gestellt werden, ob rein technischer Fortschritt in der Erkennung von Deepfakes ausreicht, um ausreichend dagegen vorgehen zu können.
\subsection{Alternative Hypothese und unterstützende Argumente}
Diese Arbeit verfolgt das Ziel zu zeigen, dass rein technischer Fortschritt in der Deepfake Detection speziell bei Audio Deepfakes nicht ausreichend ist.
Dazu wird zunächst der gesellschaftliche Aspekt beleuchtet, speziell bereits durchgeführte Studien zum Faktor Mensch.
Dabei kann festgestellt werden, dass der Addressat von Deepfakes zugleich eine oder vielleicht sogar die größte Schwachstelle bei der Detektion ist.
Kaum jemand ist, besonders bei gut gemachten und richtig plazierten Audio Deepfakes, in der Lage diese sofort zu erkennen.
Zusammen mit der Natur dieser Fakes, häufig in Echtzeit verwendet oder verbreitet zu werden (Telefonbetrug oder Verbreitung von gefälschten Audioaufnahmen über Soziale Medien) ist genau das ein großes Problem.

Technisch können aufgezeichnete Deepfakes oder gefakte Aufzeichnungen zwar häufig identifiziert werden, allerdings mit beliebigem zeitlichen oder geldlichen Aufwand.
Aufgrund der Vielzahl an verfügbaren Übersichtsartikeln über Detection Mechanismen für Audio Deepfakes können hier mehrere Gründe angeführt werden.
Beispielsweise ist es bei den meisten Mechanismen nötig den verwendeten Erzeugungsmechanismus zu kennen.
Es müssen also ständig neue Erzeugungsmechanismen analysiert und Datensätze erzeugt werden mit denen ein Detection Mechanismus dann lernen kann diese Art von Audio Deepfake zu erkennen.

Die Kombination von Echtzeitverbreitung von (Fehl-)Informationen mit dem Instinkt von Menschen optischen und akustischen Informationen eher unkritisch gegenüberzustehen ist eine riesengroße Herausforderung.
Bisher ist es zwar mit teilweise hohem forensischen Aufwand möglich die meisten Fakes zu entlarven, jedoch lassen sich keine Artikel finden die eine Echtzeitanalyse von Audio Deepfakes in Aussicht stellen.
Daher sind zusätzlich zur weiteren Erforschung von Erzeugungs- und Detection Mechanismen weitere Faktoren bei der Bekämpfung von Audio Deepfakes erforderlich.
Einen kleinen Überblick über bereits dokumentierte aber möglicherweise nicht genug beachtete Vorschläge gibt dieser Artikel.

\section{Fazit}
Anwendung von Audio Deepfake findet (aufgrund der medialen Natur) hauptsächlich in Echtzeit statt (Telefonscam).
Verbreitung durch soziale Medien auch.
Dadurch ist das Gefahrenpotential entsprechend groß (wichtigste Punkte anführen).
Herausforderung, die Erkennung ebenfalls in Echtzeit zu realisieren ist riesengroß (kurz nochmal auf Gründe eingehen).

Alternative technische aber auch gesellschaftliche Lösungen präsentieren.
Digitaler Fingerabdruck, Zensierung durch Anbieter von sozialen Medien, etc. mit ganz kurzen pro/con.
Aufklären! Wie bei sozialen Medien in Grundschule oder so anfangen: nicht alles glauben etc.

Das führt zum Abschluss:

Wir Leben in „post-truth era“, müssen lernen audio-visuelle Medien kritisch zu betrachten.
Bisher wurde ein Video/O-Ton als Beweis verwendet: nicht mehr möglich.
% \section{Misc}
% Einstiegshürde bezogen auf die Erzeugung von Audio Deepfakes niedrig: Es gibt open source cross platform software zur Audiobearbeitung mit der man Deepfakes erstellen kann (inkl. Tutorials?).
% Bsp.: Audacity, Soundforge
% Passt zu Einleitung oder Gefahrenanalyse.

\appendix

%TODO Publisher bei Quellen nennen
\bibliographystyle{Literatur/Rhein_Waal}
\bibliography{Literatur/Literatur}
\end{document}