\subsection{Methodische Ansätze}
% Hier verfügbare Methoden kategorisieren und allgemein vorstellen.
% Welche Ansätze gibt es?
% Welche Voraussetzungen und Anforderungen haben die (Ressourcen, Quelle)?
In den letzten Jahren ist eine Häufung an Übersichtsartikeln über Detektionsalgorithmen für Audio Deepfakes in der Fachliteratur zu beobachten \citep[][]{Masood2022,Almutairi2022,Khanjani2021}.
Tatsächlich sind es die ersten Übersichtsartikel zu diesem Thema, der älteste wurde 20XX veröffentlicht. %TODO \citep[][]{ältestes Review}
Das zeigt wie frisch und hochaktuell die wissenschaftliche Auseinandersetzung mit der Thematik Audio Deepfake ist.
Außerdem fällt auf, dass viele unterschiedliche Ansätze von verschiedenen Forschungsgruppen zu finden sind.
Hier scheint sich eine Community zu finden, die sich einen ersten Überblick über die unterschiedlichen Ausprägungen ihrer Forschung verschaffen will.

Diesen Überblick wollen wir nutzen, um zu untersuchen inwieweit die aktuell erforschten Detektionsalgorithmen bei der zeitkritischen Bekämpfung von Audio Deepfakes unterstützen können.
Dazu werden zunächst die grundsätzlichen Methoden zur Detektion von Audio Deepfakes vorgestellt.
Anschließend werden diese hinsichtlich der Voraussetzungen und Anforderungen an benötigte Ressourcen sowie die Audioquelle selbst analysiert.
Dabei stellt sich stets die Frage, ob die Technologie bereits jetzt oder eventuell in naher Zukunft prinzipiell einen Telefonbetrug schnell genug aufdecken kann, um diesen verhindern zu können.

\textbf{Text-to-speech TTS} \citep[][]{Masood2022}
\begin{itemize}
  \item deep neural network
  \item WaveNet, Tacotron, DeepVoice3
  \item konvertiert Text in Sprache, basierend auf Sprachbeispielen des Zielsprechers
  \item Spezialfall: Voice Cloning, in Echtzeit, basierend auf Sprachbeispielen
  \item $\rightarrow$ online Platformen mit öffentlicher Verfügbarkeit (Overdub, VoiceApp, iSpeech)
  \item Sprachcharakteristika schlecht abgebildet (Emotionen, Ausdruck, Stress, Atmung, Akzent)
\end{itemize}

\textbf{Voice Conversion}
\begin{itemize}
  \item 
\end{itemize}