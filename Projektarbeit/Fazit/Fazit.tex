\section{Fazit}
Anwendung von Audio Deepfake findet (aufgrund der medialen Natur) hauptsächlich in Echtzeit statt (Telefonscam).
Verbreitung durch soziale Medien auch.
Dadurch ist das Gefahrenpotential entsprechend groß (wichtigste Punkte anführen).
Herausforderung, die Erkennung ebenfalls in Echtzeit zu realisieren ist riesengroß (kurz nochmal auf Gründe eingehen).

Alternative technische aber auch gesellschaftliche Lösungen präsentieren.
Digitaler Fingerabdruck, Zensierung durch Anbieter von sozialen Medien, etc. mit ganz kurzen pro/con.
Aufklären! Wie bei sozialen Medien in Grundschule oder so anfangen: nicht alles glauben etc.

Das führt zum Abschluss:

Wir Leben in „post-truth era“, müssen lernen audio-visuelle Medien kritisch zu betrachten.
Bisher wurde ein Video/O-Ton als Beweis verwendet: nicht mehr möglich.
Glaubwürdigkeit leidet, Menschen können schwieriger zwischen Wahrheit und Lüge unterscheiden \citep[][]{Godulla2021}.