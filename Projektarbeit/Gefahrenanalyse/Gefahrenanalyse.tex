\section{Gefahrenanalyse}
Gesellschaftliche Betrachtung der Thematik.
2-3 Sätze die erklären, was im Kapitel passiert.
\subsection{Gefahrenpotential allgemein}
Hier zunächst allgemeine Gefahrenpotentiale von gezielter Fehlinformation definieren.
\subsection{Konkrete Beispiele beschreiben}
Gefahrenpotentiale an Beispielen konkretisieren: \textbf{Was} wurde gemacht um \textbf{was} zu erreichen?
Hier Fokus auf Echtzeitproblematik legen? Oder folgt das aus dem nächsten Unterkapitel?
\subsection{Studien zur menschlichen Wahrnehmung}
Analyse von durchgeführten Studien: Inwieweit ist der Mensch anfällig für Deepfakes?
Welche Konsequenz ergibt sich daraus? (Anforderungen an Informationen, Detektion von DF, etc.)