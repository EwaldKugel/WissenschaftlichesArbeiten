\section{Gefahrenanalyse}
% Gesellschaftliche Betrachtung der Thematik.
% 2-3 Sätze die erklären, was im Kapitel passiert.
In diesem Kapitel werden die Angriffspunkte von Deepfakes und den damit verbundenen Gefahren für die Gesellschaft beschrieben.
Anhand einiger Beispiele wird aufgezeigt, welchen Einfluss Deepfakes haben, um beim Menschen durch gezielte Manipulation Schaden anzurichten.
Weiterhin werden Möglichkeiten zur schnelleren Erkennung solcher Manipulationen diskutiert.
\subsection{Gefahrenpotential allgemein}
% Hier zunächst allgemeine Gefahrenpotentiale von gezielter Fehlinformation definieren.
Das Konzept der Falschinformation, in der heutigen Zeit oft unter dem Begriff ``Fake News'' genannt, ist eine Methode die bereits seit mehr als 125 Jahren eingesetzt wird um Menschen gezielt zu manipulieren.
Gerade im Zeitalter der sozialen Medien, in denen sich Menschen öfter dieser Meldungen bedienen als konventioneller Nachrichten, wächst die Bedrohung durch Falschinformation täglich (\cite{Lee2019}).
Eine große Gefahr die dabei von Deepfakes ausgeht, ist dass jeder die Möglichkeit hat, diese mit kostenfreier Software zu erstellen (\cite{Appel2022}).
Das gefährliche dabei ist die millionenfache Verbreitung innerhalb Sekunden um die ganze Welt, in der Jeder Ziel eines solchen Angriffs sein kann (\cite{Shahzad2022}).
\par
Dabei reichen die Angriffsziele von privaten Personen, über Prominente Personen bis hin zu ganzen Staaten.
Besonders letzteres ist ein immer häufiger gewähltes Ziel, gerade bei politschen Wahlen oder Zwischenstaatlichen Beziehungen.
So kursierte im März 2022 in sozialen Medien ein Deepfake-Video des Ukrainischen Präsidenten Volodymyr Zelenskyy, in dem er im Rahmen des Russisch-ukrainischen Krieges die Kapitulation seiner Soldaten forderte.
Einmal durch die Kanäle sozialer Medien verbreitet, lassen sich diese Falschmeldungen schwierig Löschen oder Richtigstellen und wenn doch, dann ist der Schaden bereits angerichtet.
Soziale Netzwerke oder Messenger wie Facebook, WhatsApp oder Telegram sind dabei besonders kritisch zu betrachten, da sie die Verbreitung von ungeprüften Inhalten jeglicher Art ermöglichen.
\par
Deepfakes erweitern und verschlimmern also das bereits bekannte Phänomen der ``Fake News'', indem sie sich der gleichen Kanäle bedienen, die Inhalte allerdings noch deutlich realistischer und glaubhafter darstellen (\cite{Appel2022}).
Gerade das hohe Level von Faktoren wie Verbreitungsgeschwindigkeit, Realismus und Personalisierung machen Deepfakes zu einer immer weiter wachsenden Gefahr für die Gesellschaft.
Was den Menschen dabei so angreifbar macht, ist die Tatsache, dass vermeintlich selbst wahrgenommen Inhalte wie Videos und Fotos, einen stärkeren Eindruck hinterlassen als niedergschriebener Text.
Besonders wenn es sich dabei um vertraute Personen und deren Auftreten oder Stimme handelt (\cite{Kietzmann2020}).
\subsection{Konkrete Beispiele beschreiben}
Gefahrenpotentiale an Beispielen konkretisieren: \textbf{Was} wurde gemacht um \textbf{was} zu erreichen?
Hier Fokus auf Echtzeitproblematik legen? Oder folgt das aus dem nächsten Unterkapitel?
\subsection{Studien zur menschlichen Wahrnehmung}
Analyse von durchgeführten Studien: Inwieweit ist der Mensch anfällig für Deepfakes?
Welche Konsequenz ergibt sich daraus? (Anforderungen an Informationen, Detektion von DF, etc.)