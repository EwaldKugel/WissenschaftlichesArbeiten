\subsection{Aussagen zur Forschung}
Die Forschung, die auf diesem Gebiet betrieben wird, setzt sich häufig mit der Erstellung und Erkennung von diesen Deepfakes auseinander.
Bei der Untersuchung möglicher Identifikatoren wird zwischen der Erkennung durch den Menschen und Erkennung von Algorithmen unterschieden.
Die aktuelle Aussicht ist die, dass die Technologie zur Erkennung sich stark verbessern wird, die menschlichen Fähigkeiten allerdings stagnieren (\cite{Mueller2022}).
Hauptproblem bei der Erkennung von Deepfakes ist die zeitliche Komponente.
Gerade über soziale Medien erreichen Deepfakes innerhalb von Sekunden Millionen Menschen.
Studien zeigen potentielle Maßnahme zur Bekämpfung von Deepfakes wie politsche Richtlinien, Sensibilisierung, Schulung sowie Bildung (\cite{Shahzad2022}).
Auch wird der Einsatz von technischen Forensikern in sensiblen Gebieten wie vor Gericht empfohlen (\cite{Jones2022}).
Diese Technologie steht erst am Anfang und jeder ist in der Lage mit einfachster Software Deepfakes zu erstellen.
Der Fokus liegt zwischen der Erstellung und Erkennung von Deepfakes, der ``Kalte Krieg'' zwischen diesen beiden hat gerade erst begonnen (\cite{Masood2022})