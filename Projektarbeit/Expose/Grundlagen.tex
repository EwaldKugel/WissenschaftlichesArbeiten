\subsection{Grundlagen}
Bei Deepfakes handelt es sich um manipulierte Audio, Video und Bilddateien, die mit Hilfe von Deep Learning, also KI, erstellt werden.
Es gibt einige Beispiele zur Anwendung von Deepfakes, wie eine Videomanipulation, in der Barack Obama Donald Trump ``a total and complete dipshit'' nennt.
Gerade im politischen Sektor können Deepfakes genutzt werden um einen Gegner zu diskreditieren.
Diese Möglichkeit der politischen Dikreditierung zeigt eine Studie, in der die Auswirkungen auf eine manipulierte Videodatei eines holländischen Politikers untersucht werden.
Darüberhinaus zeigt die Studie, dass die Verwendung von Deepfakes eine höheren Einfluss hat als die Verwendung von klassischen Falschmeldungen (\cite{Dobber2020}).
Dieses Paper beschäftigt sich allerdings mit der Gefahr, die von Audio-Deepfakes ausgeht.
Es gibt bekannte Fälle, in denen sich sogenannte Scammer als CEO einer Firma ausgeben und als dieser eine Summe Geld einfordern (cite Onlinequelle Fake Johannes).
Der Schwerpunkt des Papers liegt bei der Gefahr, die von Audio-Deepfakes ausgeht, sowie die optimierte Erkennung von diesen.
