\subsection{Alternative Hypothese und unterstützende Argumente}
Diese Arbeit verfolgt das Ziel zu zeigen, dass rein technischer Fortschritt in der Deepfake Detection speziell bei Audio Deepfakes nicht ausreichend ist.
Dazu wird zunächst der gesellschaftliche Aspekt beleuchtet, speziell bereits durchgeführte Studien zum Faktor Mensch.
Dabei kann festgestellt werden, dass der Addressat von Deepfakes zugleich eine oder vielleicht sogar die größte Schwachstelle bei der Detektion ist.
Kaum jemand ist, besonders bei gut gemachten und richtig plazierten Audio Deepfakes, in der Lage diese sofort zu erkennen.
Zusammen mit der Natur dieser Fakes, häufig in Echtzeit verwendet oder verbreitet zu werden (Telefonbetrug oder Verbreitung von gefälschten Audioaufnahmen über Soziale Medien) ist genau das ein großes Problem.

Technisch können aufgezeichnete Deepfakes oder gefakte Aufzeichnungen zwar häufig identifiziert werden, allerdings mit beliebigem zeitlichen oder geldlichen Aufwand.
Aufgrund der Vielzahl an verfügbaren Übersichtsartikeln über Detection Mechanismen für Audio Deepfakes können hier mehrere Gründe angeführt werden.
Beispielsweise ist es bei den meisten Mechanismen nötig den verwendeten Erzeugungsmechanismus zu kennen.
Es müssen also ständig neue Erzeugungsmechanismen analysiert und Datensätze erzeugt werden mit denen ein Detection Mechanismus dann lernen kann diese Art von Audio Deepfake zu erkennen.

Die Kombination von Echtzeitverbreitung von (Fehl-)Informationen mit dem Instinkt von Menschen optischen und akustischen Informationen eher unkritisch gegenüberzustehen ist eine riesengroße Herausforderung.
Bisher ist es zwar mit teilweise hohem forensischen Aufwand möglich die meisten Fakes zu entlarven, jedoch lassen sich keine Artikel finden die eine Echtzeitanalyse von Audio Deepfakes in Aussicht stellen.
Daher sind zusätzlich zur weiteren Erforschung von Erzeugungs- und Detection Mechanismen weitere Faktoren bei der Bekämpfung von Audio Deepfakes erforderlich.
Einen kleinen Überblick über bereits dokumentierte aber möglicherweise nicht genug beachtete Vorschläge gibt dieser Artikel.