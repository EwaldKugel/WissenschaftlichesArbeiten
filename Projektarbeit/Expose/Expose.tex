\section{Expose}



\subsection{Grundlagen}
Bei Deepfakes handelt es sich um manipulierte Audio, Video und Bilddateien, die mit Hilfe von Deep Learning, also KI, erstellt werden.
Es gibt einige Beispiele zur Anwendung von Deepfakes, wie eine Videomanipulation, in der Barack Obama Donald Trump ``a total and complete dipshit'' nennt.
Gerade im politischen Sektor können Deepfakes genutzt werden um einen Gegner zu diskreditieren.
Diese Möglichkeit der politischen Dikreditierung zeigt eine Studie, in der die Auswirkungen auf eine manipulierte Videodatei eines holländischen Politikers untersucht werden.
Darüberhinaus zeigt die Studie, dass die Verwendung von Deepfakes eine höheren Einfluss hat als die Verwendung von klassischen Falschmeldungen (\cite{Dobber2020}).
Dieses Paper beschäftigt sich allerdings mit der Gefahr, die von Audio-Deepfakes ausgeht.
Es gibt bekannte Fälle, in denen sich sogenannte Scammer als CEO einer Firma ausgeben und als dieser eine Summe Geld einfordern (cite Onlinequelle Fake Johannes).
Der Schwerpunkt des Papers liegt bei der Gefahr, die von Audio-Deepfakes ausgeht, sowie die optimierte Erkennung von diesen.
\subsection{Aussagen zur Forschung}
Die Forschung, die auf diesem Gebiet betrieben wird, setzt sich häufig mit der Erstellung und Erkennung von diesen Deepfakes auseinander.
Bei der Untersuchung möglicher Identifikatoren wird zwischen der Erkennung durch den Menschen und Erkennung von Algorithmen unterschieden.
Die aktuelle Aussicht ist die, dass die Technologie zur Erkennung sich stark verbessern wird, die menschlichen Fähigkeiten allerdings stagnieren (\cite{Mueller2022}).
Hauptproblem bei der Erkennung von Deepfakes ist die zeitliche Komponente.
Gerade über soziale Medien erreichen Deepfakes innerhalb von Sekunden Millionen Menschen.
Studien zeigen potentielle Maßnahme zur Bekämpfung von Deepfakes wie politsche Richtlinien, Sensibilisierung, Schulung sowie Bildung (\cite{Shahzad2022}).
Auch wird der Einsatz von technischen Forensikern in sensiblen Gebieten wie vor Gericht empfohlen (\cite{Jones2022}).
Diese Technologie steht erst am Anfang und jeder ist in der Lage mit einfachster Software Deepfakes zu erstellen.
Der Fokus liegt zwischen der Erstellung und Erkennung von Deepfakes, der ``Kalte Krieg'' zwischen diesen beiden hat gerade erst begonnen (\cite{Masood2022})
